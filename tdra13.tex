\documentclass[twocolumn,UTF8]{ctexart}
\usepackage[mtphrb]{mtpro2}%,amssymb
\ctexset{section = {name = {,、\hspace*{-5mm}},number = \chinese{section},format = {\biaosong\zihao{-4}\bfseries\raggedright}}}
\usepackage{titlesec}%titlesec宏包调整section与正文间距
\titlespacing*{\section} {0pt}{9pt}{4pt}
%===============================================================================%
\setCJKmainfont{Adobe Song Std L}%中文默认字体:adobe 宋体
\renewcommand{\heiti}{\CJKfontspec{Adobe Heiti Std R}}%adobe 黑体
%\renewcommand{\heiti}{\CJKfontspec{Hiragino Sans GB}}%冬青黑体简体中文
\renewcommand{\fangsong}{\CJKfontspec{Adobe Fangsong Std R}}%adobe 仿宋
\renewcommand{\kaishu}{\CJKfontspec{Adobe Kaiti Std R}}%adobe 楷体
\newcommand*{\xingkai}{\CJKfamily{STxingkai}}
%\newcommand{\zhongsong}{\CJKfontspec{STZhongsong}}
\newcommand{\biaosong}{\CJKfontspec{方正小标宋简体}}%方正粗宋简体
%===============================================================================%
\usepackage{zref-user,zref-lastpage}%使用zref宏包,引用数字标签值和LastPage标签,感谢qingkuan大神指导
\usepackage{times} %use the Times New Roman fonts
\usepackage{bigstrut}
\usepackage{enumerate}
\usepackage{amsmath,bm}
\everymath{\displaystyle}
\newcommand\dif{\mathop{}\!\mathrm{d}}
\usepackage[paperwidth=42cm,paperheight=29.7cm,top=4.4cm,bottom=2.5cm,left=4.5cm,right=1cm]{geometry}
\usepackage{fancyhdr}\pagestyle{fancy}
\renewcommand{\headrulewidth}{0pt}
\renewcommand{\footrulewidth}{0pt}
%%%%%%%%%%%%%%%%%%%%%%%%%%%%%%%%%%%%%%%%%%%%%%%%%%%%%%%%%%%%%%%%%%%%%%%
\usepackage{tikz}
\usepackage{fancybox}
\fancyput(1.50cm,-25.3cm){\tikz \draw[solid,line width=2pt](0,0) rectangle (1cm+\textwidth,1.2cm+\textheight);}    
 %solid,dashed%pdfmanual.pdf---p167
%%%%%%%%%%%%%%%%%%%%%%%%%%%%%%%%%%%%%%%%%%%%%%%%%%%%%%%%%%%%%%%%%%%%%%%
%%%%%%%%%%%%%%%%%%%%%%%%%%%%%%%%%%%选择题%%%%%%%%%%%%%%%%%%%%%%%%%%%%%%%%%%%%%%%%
%选项单行
\newcommand{\xo}[4]{\makebox[100pt][l]{(A) #1} \hfill
                    \makebox[100pt][l]{(B) #2} \hfill
                    \makebox[100pt][l]{(C) #3} \hfill
                    \makebox[100pt][l]{(D) #4}}
%选项分两行。
\newcommand{\xab}[2]{\makebox[100pt][l]{(A) #1} \hfill
                     \makebox[220pt][l]{(B) #2}}
\newcommand{\xcd}[2]{\makebox[100pt][l]{(C) #1} \hfill
                     \makebox[220pt][l]{(D) #2}}
%选项分四行.
\newcommand{\xa}[1]{(A) #1}
\newcommand{\xb}[1]{(B) #1}
\newcommand{\xc}[1]{(C) #1}
\newcommand{\xd}[1]{(D) #1}
%%%%%%%%%%%%%%%%%%%%%%%%%%%%%%%%%%%%%%%%%%%%%%%%%%%%%%%%%%%%%%%%%%%%%%%%
\textwidth=34.6cm        %文本的宽度

\begin{document}\zihao{-4}
\setlength{\columnseprule}{0pt}
\renewcommand\arraystretch{1.5}	
\fancyhead[LO,LE]{\zihao{4}\vspace*{-18mm}\hspace{-4mm}{\heiti 系别}\underline{\hspace{1.5cm}}{\heiti 专业}\underline{\hspace{3.5cm}}\hspace{1cm}\underline{\hspace{1.2cm}}{\heiti 班}}
\fancyhead[CO,CE]{\vspace*{-18mm}{\setlength{\unitlength}{4mm}\begin{picture}(15,0)\put(-3,2.5){\zihao{-2}天津大学仁爱学院专用纸}\end{picture}}\\\zihao{4}{\heiti 年级}\underline{\hspace{2cm}}{\heiti 学号}\underline{\hspace{4cm}}{\heiti 姓名}\underline{\hspace{32mm}}}
\fancyhead[RO,RE]{\vspace*{-18mm}\zihao{4}第\;\thepage\;页\quad\; 共\;\,\zpageref{LastPage}\; 页\hspace*{4cm}}
\cfoot{雷电法王杨永信}  
\begin{center}\vspace*{-4mm}
{\zihao{-2}\heiti 2013$\sim$2014学年第一学期期中考试试卷}\\[6mm]
{\zihao{4}\heiti《高等数学1A》\;(共\zpageref{LastPage}页)}\\[4mm]      
%输出"绝密"字样
%{\heiti 绝密$\bigstar$启用前\\[-13.5mm]%缩短"绝密"字样与总计分表之间的距离
({\zihao{-4}\heiti 考试时间: 2013年11月8日})\\
\begin{tabular}{|c|c|c|c|c|c|c|c|c|}\hline
\centering ~题号~ & \centering\hspace{2mm} 一 \hspace{2mm} & \centering \hspace{2mm} 二 \hspace{2mm} & \centering \hspace{2mm} 三 \hspace{2mm} &\centering\hspace{2mm} 四 \;\,\hspace{2mm}& \centering \hspace{2mm} 五 \hspace{2mm} & \centering \hspace{2mm} 六 \,\hspace{1.8mm}  &\centering \hspace{0.3mm} 成绩 \hspace{0.3mm} &\hspace{1mm}核分人签字\hspace{1mm} \bigstrut\\\hline
\centering ~得分 &  &  &  &  &  &   && \bigstrut\\ \hline
\end{tabular}\\[5mm]
  \end{center}
 %%%%%%%%%%%%%%%%%%%%%%%%%%%%%%%%%一、填空题%%%%%%%%%%%%%%%%%%%%%%%%%%%%%%%%%%%%%%%
\section{填空题\songti{(本题满分9分, 每小题3分)}}
 \begin{enumerate}
\item 已知 $f(x)=\begin{cases}x^2,&x\geq 0\\\frac{1}{x},&x<0\end{cases}$ , 则 $f(x+1)=$ \underline{\hspace{2.5cm}}\\[1mm]
\item $ \lim_{x\to\infty}\left(1-\frac{k}{x}\right)^{-2x} \,(k\neq0)=$ \underline{\hspace{2.5cm}}\\[-0.3cm]
\item 函数 $f(x)=x^3-3x+1$ 在区间 $[0,2]$ 上的最小值  \underline{\hspace{2.5cm}}\\[-0.6cm]
\end{enumerate}
%%%%%%%%%%%%%%%%%%%%%%%%%%%%%%%%%二、单项选择题%%%%%%%%%%%%%%%%%%%%%%%%%%%%%%%%%%%%
\section{单项选择题\songti{(本题满分9分, 每小题3分)}}
\begin{enumerate}
\item 当 $x\to\infty$ 时, 下列函数中极限存在的是 $(\hspace{0.5cm})$\\[3mm]
\xo{$\sin x$}{$e^{-\frac{1}{x}}$}{$\frac{x+1}{x^2-1}$}{$\ln|x|$}\\[-2mm]
\item 函数 $y=f(x)$ 在点 $x_0$ 处有增量 $\Delta x=0.3$,  对应函数增量的线性主要部分等于 0.9 ,\\  则 $f'(x_0)=$$(\hspace{0.5cm})$\\[3mm]
\xo{$3$}{$0.3$}{$2.7$}{$\dfrac{1}{3}$}\\[-2mm]
\item 下列函数在给定的区间上满足拉格朗日中值定理的是 $(\hspace{0.5cm})$\\[2mm]
\xab{$f(x)=|x-1|,\,[0,2]$}{$f(x)=\sqrt[3]{x},\,[-1,1]$}\\[3mm]
\xcd{$f(x)=x+|x|,\,[-1,2]$}{$f(x)=\ln(x-2),\,[3,6]$}
\end{enumerate}
%%%%%%%%%%%%%%%%%%%%%%%%%%%%%%%%三、解下列各题%%%%%%%%%%%%%%%%%%%%%%%%%%%%%%%%%%%%%
\newpage\section{解下列各题\songti{(本题满分28分, 每小题7分)}}
\begin{enumerate}
\item $ \lim\limits_{x\to0}\left(\dfrac{1}{x}-\dfrac{1}{e^x-1}\right) $\\[3.5cm]
\item 设 $y=\ln\left(x+\sqrt{x^2+1}\right)$ , 求  $y'\; y''$\\[3.5cm]
\item 设 $y=\dfrac{1}{x^2-4x+3}$ , 求  $y^{(n)}$\\[3.5cm]
\item 方程 $2^{xy}=x+y$ 确定了函数求 $y=y(x)$, 求 $y'$ 及 $\mathrm{d}y\Big|_{x=0}$ \\[3.5cm]
\end{enumerate}
%%%%%%%%%%%%%%%%%%%%%%%%%%%%%%%%四、解下列各题%%%%%%%%%%%%%%%%%%%%%%%%%%%%%%%%%%%%%
\newpage\section{解下列各题\songti{(本题满分35分, 每小题7分)}}
\begin{enumerate}
\item $\lim_{x\to0}\frac{\sin x-x}{x^2\sin x}$\\[5cm]
\item 设 $f(x)=\left\{\begin{array}{lc}\dfrac{1-\cos x}{\sqrt{x}}&x>0\\[1.8mm]
0,&x=0\\x\sin\dfrac{1}{x}&x< 0
\end{array}\right.$ , 判断 $f(x)$ 在 $x=0$ 处的连续性, \\[2mm]并求 $\lim_{x\to-\infty}f(x)$,$\lim_{x\to+\infty}f(x)$\\[7cm]
\item 参数方程 $ \begin{cases}x=\arctan t&\\y=\ln t&\end{cases} $ 确定了函数 $ y=y(x) $ , 求 $ \dfrac{\mathrm{d}y}{\mathrm{d}x} $ 及  $ \dfrac{\mathrm{d}^2y}{\mathrm{d}x^2} $
\end{enumerate}
\newpage\vspace*{4cm}\begin{enumerate}\setcounter{enumi}{3}
\item 讨论函数 $f(x)=3x^2-x^3$ 的单调性和凹凸性 , 并求极值点和拐点\\[7cm]
\item 求曲线 $y=\dfrac{\cos2x}{x}$ 的水平渐近线和铅直渐近线 \\[7cm]
\end{enumerate}
%%%%%%%%%%%%%%%%%%%%%%%%%%%%%%五、解下列各题%%%%%%%%%%%%%%%%%%%%%%%%%%%%%%%%%%%%%%%
\newpage\section{解下列各题\songti{(本题满分14分, 每小题7分)}}
\begin{enumerate}
\item 设有一根长为 $l$ 的铁丝 , 将其分为两段 , 分别构成圆形和正方形 , 若圆形的面积为 \\$S_1$, 正方形的面积为 $S_2$, 求当正方形的边长为多少时  $S_1+S_2$  取得最小值.\\[7cm]
\item 证明 : 当 $x>0$ 时 , $\ln(1+x)<x-\dfrac{x^2}{2}+\dfrac{x^3}{3}$
\end{enumerate}
%%%%%%%%%%%%%%%%%%%%%%%%%%%%%%%%%六、证明题%%%%%%%%%%%%%%%%%%%%%%%%%%%%%%%%%%%%%%%
\newpage\section{证明题\songti{(本题满分5分)}}
已知 $f(x)$  在 $[0,1]$ 内二阶可导, 且 $f(0)=f(1)=0$, $F(x)=(x-1)f(x)$\\
证明:  至少存在一个 $\xi\in(0,1)$ , 使得 $F''(\xi)=0$












\end{document}

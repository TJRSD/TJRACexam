\documentclass[twocolumn,UTF8]{ctexart}
\usepackage[mtphrb]{mtpro2}%,amssymb
\ctexset{section = {name = {,、\hspace*{-5mm}},number = \chinese{section},format = {\biaosong\zihao{-4}\raggedright}}}
\usepackage{titlesec}%titlesec宏包调整section与正文间距
\titlespacing*{\section} {0pt}{9pt}{4pt}
%===============================================================================%
\setCJKmainfont{Adobe Song Std L}%中文默认字体:adobe 宋体
\renewcommand{\heiti}{\CJKfontspec{Adobe Heiti Std R}}%adobe 黑体
%\renewcommand{\heiti}{\CJKfontspec{Hiragino Sans GB}}%冬青黑体简体中文
\renewcommand{\fangsong}{\CJKfontspec{Adobe Fangsong Std R}}%adobe 仿宋
\renewcommand{\kaishu}{\CJKfontspec{Adobe Kaiti Std R}}%adobe 楷体
\newcommand*{\xingkai}{\CJKfamily{STxingkai}} 
%\newcommand{\zhongsong}{\CJKfontspec{STZhongsong}}
\newcommand{\biaosong}{\CJKfontspec{方正小标宋简体}}%方正粗宋简体
%===============================================================================%
\usepackage{zref-user,zref-lastpage}%使用zref宏包,引用数字标签值和LastPage标签,感谢qingkuan大神指导
\usepackage{bigstrut}
\usepackage{enumerate}
\usepackage{amsmath,bm}
\everymath{\displaystyle}
\newcommand\dif{\mathop{}\!\mathrm{d}}
\usepackage{CJKnumb}%中文小写数字
\usepackage[paperwidth=42cm,paperheight=29.7cm,top=4.4cm,bottom=2.5cm,left=4.5cm,right=1cm]{geometry}
\usepackage{fancyhdr}\pagestyle{fancy}
\renewcommand{\headrulewidth}{0pt}
\renewcommand{\footrulewidth}{0pt}
%%%%%%%%%%%%%%%%%%%%%%%%%%%%%%%%%%%%%%%%%%%%%%%%%%%%%%%%%%%%%%%%%%%%%%%
\usepackage{tikz}
\usepackage{fancybox}
\fancyput(1.50cm,-25.3cm){\tikz \draw[solid,line width=2pt](0,0) rectangle (1cm+\textwidth,1.2cm+\textheight);}    
 %solid,dashed%pdfmanual.pdf---p167
%%%%%%%%%%%%%%%%%%%%%%%%%%%%%%%%%%%%%%%%%%%%%%%%%%%%%%%%%%%%%%%%%%%%%%%

%%%%%%%%%%%%%%%%%%%%%%%%%%%%%%%%%%%选择题%%%%%%%%%%%%%%%%%%%%%%%%%%%%%%%%%%%%%%%%
%选项单行
\newcommand{\xo}[4]{\makebox[100pt][l]{(A) #1} \hfill
                    \makebox[100pt][l]{(B) #2} \hfill
                    \makebox[100pt][l]{(C) #3} \hfill
                    \makebox[100pt][l]{(D) #4}}
%选项分两行。
\newcommand{\xab}[2]{\makebox[100pt][l]{(A) #1} \hfill
                     \makebox[220pt][l]{(B) #2}}
\newcommand{\xcd}[2]{\makebox[100pt][l]{(C) #1} \hfill
                     \makebox[220pt][l]{(D) #2}}
%选项分四行.
\newcommand{\xa}[1]{(A) #1}
\newcommand{\xb}[1]{(B) #1}
\newcommand{\xc}[1]{(C) #1}
\newcommand{\xd}[1]{(D) #1}
%%%%%%%%%%%%%%%%%%%%%%%%%%%%%%%%%%%%%%%%%%%%%%%%%%%%%%%%%%%%%%%%%%%%%%%%
\textwidth=34.6cm        %文本的宽度

\begin{document}\zihao{-4}
\setlength{\columnseprule}{0pt}
\renewcommand\arraystretch{1.5}
\fancyhead[LO,LE]{\zihao{4}\vspace*{-18mm}\hspace{-4mm}{\heiti 系别}\underline{\hspace{1.5cm}}{\heiti 专业}\underline{\hspace{3.5cm}}\hspace{1cm}\underline{\hspace{1.2cm}}{\heiti 班}}
\fancyhead[CO,CE]{\vspace*{-18mm}{\setlength{\unitlength}{4mm}\begin{picture}(15,0)\put(-3,2.5){\zihao{-2}天津大学仁爱学院专用纸}\end{picture}}\\\zihao{4}{\heiti 年级}\underline{\hspace{2cm}}{\heiti 学号}\underline{\hspace{4cm}}{\heiti 姓名}\underline{\hspace{32mm}}}
\fancyhead[RO,RE]{\vspace*{-18mm}\zihao{4}第\;\thepage\;页\quad\; 共\;\,\zpageref{LastPage}\; 页\hspace*{4cm}}
\cfoot{雷电法王杨永信}  
%%%%%%%%%%%%%%%%%%%%%%%%%%%%%%%%%%%%%%%%%%%%%%%%%%%%%%%%%%%%%%%%%%%%%%%%%%
\begin{center}\vspace*{-4mm}
{\zihao{-2}\heiti 2014$\sim$2015学年第一学期期中考试试卷}\\[6mm]
{\zihao{4}\heiti《高等数学3A》\;(共\zpageref{LastPage}页)}\\[4mm]
%输出"绝密"字样      
%{\heiti 绝密$\bigstar$启用前\\[-13.5mm]%缩短"绝密"字样与总计分表之间的距离
({\zihao{-4}\heiti 考试时间: 2014年11月7日})\\
\begin{tabular}{|c|c|c|c|c|c|c|c|c|}\hline
\centering ~题号~ & \centering\hspace{2mm} 一 \hspace{2mm} & \centering \hspace{2mm} 二 \hspace{2mm} & \centering \hspace{2mm} 三 \hspace{2mm} &\centering\hspace{2mm} 四 \;\,\hspace{2mm}& \centering \hspace{2mm} 五 \hspace{2mm} & \centering \hspace{2mm} 六 \,\hspace{1.8mm}  &\centering \hspace{0.3mm} 成绩 \hspace{0.3mm} &\hspace{1mm}核分人签字\hspace{1mm} \bigstrut\\\hline
\centering ~得分 &  &  &  &  &  &   && \bigstrut\\ \hline
\end{tabular}\\[5mm]
\end{center}
%%%%%%%%%%%%%%%%%%%%%%%%%%%%%%%%%一、填空题%%%%%%%%%%%%%%%%%%%%%%%%%%%%%%%%%%%%%%%
\section{填空题\songti{(本题满分9分, 每小题3分)}}
 \begin{enumerate}
\item 函数 $f(x)=\arcsin(x-2)+\dfrac{1}{\sqrt{x^2-4}}$ 的定义域为 \underline{\hspace{2.5cm}}\\[-0.3cm]
\item 设 $y=\dfrac{\tan x}{x}$ ,  则 $\mathrm{d}y=$ \underline{\hspace{2.5cm}}\\[-0.3cm]
\item 函数 $f(x)=\ln(1-x)$ 在区间 $[-1,0]$ 上满足拉格朗日中值定理的 $\xi=$ \underline{\hspace{2.5cm}}\\[-0.6cm]
\end{enumerate}
%%%%%%%%%%%%%%%%%%%%%%%%%%%%%%%%%二、单项选择题%%%%%%%%%%%%%%%%%%%%%%%%%%%%%%%%%%%%
\section{单项选择题\songti{(本题满分9分, 每小题3分)}}
\begin{enumerate}
\item 计算极限 $\lim\limits_{x\to\infty}x\sin\dfrac{2}{x}=$ $(\hspace{0.5cm})$\\[3mm]
\xo{0}{$\dfrac{1}{2}$}{2}{$\infty$}\\[-2mm]
\item 已知  $\lim\limits_{x\to0}\dfrac{x}{f(1)-f(1-3x)}=1$  , 则 $f'(1)=$ $(\hspace{0.5cm})$\\[3mm]
\xo{$-\dfrac{1}{3}$}{$\dfrac{1}{3}$}{-3}{3}\\[-2mm]
\item 设 $f(x)$ 在 $x_0$ 处可导,  则下列各式中 \underline{不正确}的是 $(\hspace{0.5cm})$\\[2mm]
\xab{$\lim\limits_{\Delta x\to0}f(x_0+\Delta x)=f(x_0)$}{${f'_{+}}(x_0)={f'_{-}}(x_0)$}\\[3mm]
\xcd{$\mathrm{d}y\big|_{x=x_0}=f'(x_0)\mathrm{d}x$}{$\left(f(f(x))\right)'\big|_{x=x_0}=f'(f(x))\big|_{x=x_0}$}
\end{enumerate}
%%%%%%%%%%%%%%%%%%%%%%%%%%%%%%%%三、解下列各题%%%%%%%%%%%%%%%%%%%%%%%%%%%%%%%%%%%%%
\newpage\section{解下列各题\songti{(本题满分28分, 每小题7分)}}
\begin{enumerate}
\item $\lim\limits_{x\to\infty}\left(\dfrac{x^2}{x+1}-\dfrac{x^2}{x-1}\right)$\\[3.5cm]
\item 设 $y=\dfrac{x}{\sqrt{1-x^2}}$ 求  $y'\; y''$\\[3.5cm]
\item 设  $y=y(x)$  由方程$xy^2+e^{x+y}=1$  所确定的隐函数, 求 $y'$ 及 $\mathrm{d}y\big|_{x=0}$\\[3.5cm]
\item 设 $y=y(x)$ 由参数方程 $\begin{cases}
x=1-\cos t\\y=t-\sin t
\end{cases}$ 所确定的函数, 求 $\dfrac{\mathrm{d}y}{\mathrm{d}x}$ 及 $\dfrac{\mathrm{d}^2y}{\mathrm{d}x^2}$\\[3.5cm]
\end{enumerate}
%%%%%%%%%%%%%%%%%%%%%%%%%%%%%%%%四、解下列各题%%%%%%%%%%%%%%%%%%%%%%%%%%%%%%%%%%%%%
\newpage\section{解下列各题\songti{(本题满分35分, 每小题7分)}}
\begin{enumerate}
\item $\lim_{x\to0}\frac{1-\cos x}{\sin x\ln(1-2x)}$\\[5cm]
\item 设 $f(x)=\left\{\begin{array}{lc}\dfrac{e^x-1}{x}&x<0\\[1.5mm]
\biggl(1+\dfrac{2}{x+1}\biggl)^{x+1}&x\geq 0
\end{array}\right.$ , 讨论 $x=0$ 处的连续性, 若不连续判断其间断点\\[2mm]
类型并求 $\lim_{x\to+\infty}f(x)$\\[7cm]
\item $y=\ln(\sqrt{x^2+1}-x)$,  求 $y'$
\end{enumerate}
\newpage\vspace*{4cm}\begin{enumerate}\setcounter{enumi}{3}
\item 设 $f(x)=\arcsin\sqrt{x}+e^{x^2}$ 求  $f'(x)$\\[7cm]
\item 求 $y=x^3$ 在点 $(1,1)$ 处的切线方程与法线方程\\[7cm]
\end{enumerate}
%%%%%%%%%%%%%%%%%%%%%%%%%%%%%%五、解下列各题%%%%%%%%%%%%%%%%%%%%%%%%%%%%%%%%%%%%%%%
\newpage\section{解下列各题\songti{(本题满分14分, 每小题7分)}}
\begin{enumerate}
\item $\lim_{x\to0}(\cos x)^{\frac{1}{x^2}}$\\[7cm]
\item 某产品的经济函数为 $L=L(Q)=600Q-Q^3$, 求边际利润及产量为 200 时的边际利润\\并解释其经济含义
\end{enumerate}
%%%%%%%%%%%%%%%%%%%%%%%%%%%%%%%%%六、证明题%%%%%%%%%%%%%%%%%%%%%%%%%%%%%%%%%%%%%%%
\newpage\section{证明题\songti{(本题满分5分)}}
设 $f(x)$ , $g(x)$ 在 $[a,b]$ 上连续, 在 $(a,b)$ 可导, 且 $g(x)\neq0$, $f(a)g(b)=f(b)g(a)$\\
证明在 $(a,b)$ 至少存在一点 $\xi$ , 使 $f'(\xi)g(\xi)=f(\xi)g'(\xi)$












\end{document}
